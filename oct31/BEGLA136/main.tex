\documentclass[a4paper,12pt]{article}

\usepackage[utf8]{inputenc}
\usepackage[T1]{fontenc}
\usepackage{lmodern}
\usepackage{tikz}
\usetikzlibrary{positioning, arrows.meta, fit, shadows, backgrounds}
\usepackage{geometry}
\geometry{a4paper, margin=0.5in}
\usepackage{graphicx}
\usepackage{parskip}
\usepackage{enumitem}
\usepackage{hyperref}
\usepackage{times}
\usepackage{microtype}
\usepackage{eso-pic}
\usepackage[absolute,overlay]{textpos}

\begin{document}

\newpage
\mbox{}
\AddToShipoutPictureBG*{%
  \AtPageLowerLeft{
    \put(0,0){\includegraphics[width=\paperwidth,height=\paperheight]{background.jpg}}
  }%
}

\thispagestyle{empty}

\setlength{\TPHorizModule}{1mm}
\setlength{\TPVertModule}{1mm}

\begin{textblock*}{50mm}(112mm, 29mm)
  \textbf{December}
\end{textblock*}

\begin{textblock*}{50mm}(145mm, 29mm)
  \textbf{2025}
\end{textblock*}

\begin{textblock*}{100mm}(95mm, 45mm)
  \textbf{Mark L F Tlau}
\end{textblock*}

\begin{textblock*}{100mm}(95mm, 55mm)
  \textbf{2550670762}
\end{textblock*}

\begin{textblock*}{100mm}(95mm, 66mm)
  \textbf{BCA\_NEW}
\end{textblock*}

\begin{textblock*}{100mm}(95mm, 76mm)
  \textbf{BEGLA136}
\end{textblock*}

\begin{textblock*}{100mm}(95mm, 85mm)
  \textbf{07162P}
\end{textblock*}

\begin{textblock*}{100mm}(95mm, 95mm)
  \textbf{Mohyal Education and Research Institute, Qutab Institutional Area, New Delhi}
\end{textblock*}

\begin{textblock*}{100mm}(95mm, 109mm)
  \textbf{+916009341754}
\end{textblock*}

\begin{textblock*}{100mm}(95mm, 119mm)
  \textbf{wintersunset95@gmail.com}
\end{textblock*}

\begin{textblock*}{100mm}(122mm, 144mm)
  \textbf{(Yes)}
\end{textblock*}

\clearpage

\section*{SECTION A}
\begin{enumerate}
    \item \textbf{Write short notes on the following:}
    \begin{enumerate}
        \item \textbf{Difference between a portfolio and resume} \\
        \llap{\textbf{Answer:}}
        A \textbf{resume} is a concise, one-to-two-page text document that summarizes a person's professional experience,
        skills, and education. Its purpose is to quickly tell an employer if a candidate is qualified for a job. A \textbf{portfolio},
        in contrast, is a curated collection of evidence that proves those skills. It is a showcase of a person's best work,
        such as code projects, design mockups, or writing samples. In short: a resume is a summary that claims your skills,
        while a portfolio is a detailed collection that proves them.

        \item \textbf{Common Gender Perceptions} \\
        \llap{\textbf{Answer:}}
        \textbf{Common gender perceptions}, or gender stereotypes, are oversimplified and widely held beliefs about the characteristics,
        roles, and behaviors that are considered appropriate for men and women. For example, a common perception is that men
        are inherently more assertive, analytical, and suited for leadership roles (like a CEO or engineer), while women are
        perceived as being more nurturing, empathetic, and suited for support or caregiving roles (like a nurse or secretary).
        These perceptions are often limiting and can lead to bias and discrimination in the workplace.

        \item \textbf{Workplace Etiquette} \\
        \llap{\textbf{Answer:}}
        \textbf{Workplace etiquette} refers to the set of unwritten, professional rules of conduct that govern social interactions
        and behavior in a professional environment. It is about being respectful, courteous, and considerate of one's colleagues,
        superiors, and clients. Key examples include: being punctual for meetings, communicating clearly and politely (both
        in-person and via email), respecting others' time and personal space, dressing appropriately for the office, and maintaining
        a clean and organized shared workspace. Good etiquette contributes to a positive, respectful, and productive work
        environment.

        \item \textbf{Importance of Good Customer Service} \\
        \llap{\textbf{Answer:}}
        \textbf{Good customer service} is crucial for any business as it directly impacts customer loyalty and profitability.
        Its importance lies in retaining customers; it is far more cost-effective to keep an existing customer happy than
        to acquire a new one. A positive service experience builds trust and a strong brand reputation, leading to repeat
        business. Furthermore, satisfied customers are the best form of marketing—they provide positive reviews and word-of-mouth
        recommendations, which attract new customers. Conversely, poor service can quickly damage a company's reputation and
        drive customers to competitors.
    \end{enumerate}
\end{enumerate}

\clearpage

\section*{SECTION B}
\begin{enumerate}
    \item \textbf{Write a short paragraph describing any person from your institution/workplace.} \\
    \llap{\textbf{Answer:}}
    One of the most helpful people in my institution is Rohan, a fellow student in the computer science department. He is incredibly passionate about programming and is always the first to experiment with new technologies. What sets him apart is his patience and willingness to help others. While many of our peers can be competitive, Rohan is collaborative; he will often spend his free time in the lab, patiently walking classmates through a difficult coding problem or debugging a complex algorithm. He is both analytical and approachable, and his genuine enthusiasm for the subject makes complex topics seem less daunting for everyone.

    \item \textbf{List a few common body gestures and explain what each one communicates or signifies.} \\
    \llap{\textbf{Answer:}}
    Here are a few common body gestures and their significations:
    \begin{itemize}
        \item \textbf{Crossed Arms:} This is often a defensive posture. It can signify that a person is feeling closed-off, in disagreement, uncomfortable, or resistant to what is being said.
        \item \textbf{Direct Eye Contact:} Making and holding appropriate eye contact (without staring) typically communicates confidence, honesty, and genuine engagement in the conversation.
        \item \textbf{Fidgeting (e.g., tapping fingers, shaking foot):} These small, repetitive movements usually signify nervousness, impatience, boredom, or a desire for the current situation to end.
        \item \textbf{Nodding the Head:} This is a near-universal gesture of affirmation. It communicates agreement, understanding, and encourages the speaker to continue.
        \item \textbf{Avoiding Eye Contact:} Looking away, especially downwards, can signify shyness, discomfort, a lack of confidence, or in some cases, dishonesty.
    \end{itemize}

    \item \textbf{Make a presentation for your organization on sustainable workplace practices.} \\
    \llap{\textbf{Answer:}}
    Since I cannot paste a presentation into a PDF, I will write the contents of the presentation on this page.
    
    \textbf{Slide 1: Title}
    \begin{itemize}
        \item Sustainable Workplace Practices: Good for the Planet, Good for Business
    \end{itemize}
    
    \textbf{Slide 2: What is Sustainability?}
    \begin{itemize}
        \item A sustainable practice is one that meets our present needs without compromising the ability of future generations to meet theirs.
        \item It rests on three pillars: Environmental (Planet), Social (People), and Economic (Profit).
    \end{itemize}
    
    \textbf{Slide 3: Why Should We Care?}
    \begin{itemize}
        \item \textbf{Reduces Costs:} Being sustainable means being efficient. Less energy and material use equals lower utility and supply bills.
        \item \textbf{Improves Brand Reputation:} Customers and clients increasingly prefer to work with eco-conscious and socially responsible companies.
        \item \textbf{Attracts \& Retains Talent:} Modern employees, especially younger generations, want to work for organizations that align with their values.
    \end{itemize}
    
    \textbf{Slide 4: Simple Practices We Can Start Today}
    \begin{itemize}
        \item \textbf{The 3 R's (Reduce, Reuse, Recycle):} Go paperless by defaulting to digital documents. Use reusable coffee mugs and water bottles instead of disposable ones. Make sure recycling bins are clearly marked and accessible.
        \item \textbf{Energy Conservation:} Turn off lights and monitors when not in use. This is the simplest, most effective habit. As a company, we will switch to LED lighting and energy-efficient appliances.
        \item \textbf{Sustainable Commuting:} Encourage carpooling, use of public transport, or hybrid/remote work schedules to reduce our collective carbon footprint from commuting.
    \end{itemize}
    
    \textbf{Slide 5: Conclusion}
    \begin{itemize}
        \item Sustainability is not a one-time goal; it's a continuous journey of making smarter choices.
        \item Our small, consistent changes will add up to a significant positive impact.
    \end{itemize}

    \item \textbf{You are interested in applying for the position of a marketing executive in a company. Write a covering
    letter for this position, showing how you are suitable for the job.} \\
    \llap{\textbf{Answer:}}
    
    Mark L F Tlau \\
    Mehrauli, Delhi \\
    6009341754 \\
    wintersunset95@gmail.com
    
    \vspace{1cm}
    
    28$^{th}$ October 2025
    
    \vspace{1cm}
    
    Hiring Manager \\
    RoadVision AI \\
    Bhikaji Cama Place, New Delhi
    
    \vspace{0.5cm}
    
    \textbf{Subject: Application for the Position of Marketing Executive}
    
    \vspace{0.5cm}
    
    Dear Hiring Manager,
    
    \par
    I am writing to express my enthusiastic interest in the Marketing Executive position at RoadVision AI, as advertised on Linkedin. With my blend of creative marketing skills, technical proficiency, and proven ability to manage digital campaigns, I am confident that I possess the qualifications necessary to contribute significantly to your team.
    
    \par
    In my previous role as a Marketing Executive Intern, I was responsible for helping the executives. I successfully assisted in developing a campaign that increased lead generation by 25\%. I am highly proficient in market research, data analysis, and using marketing automation tools to optimize campaign performance. Furthermore, my background in technology allows me to quickly master new software platforms and effectively communicate with technical teams, ensuring our marketing efforts are always well-integrated.
    
    \par
    I am passionate about RoadVision AI's mission and am eager to bring my data-driven approach and creative energy to your team. I am available for an interview at your earliest convenience and can be reached via email or phone.
    
    \par
    Thank you for your time and consideration.
    
    \vspace{0.5cm}
    
    Sincerely, \\
    Mark L F Tlau

    \item \textbf{Discuss the characteristics of work Ethics by giving suitable examples.} \\
    \llap{\textbf{Answer:}}
    Work ethics are a set of moral principles and values that guide an individual's behavior in their professional life. A strong work ethic is built on several key characteristics. \textbf{Integrity} is paramount; this is the quality of being honest and having strong moral principles. For example, an employee with integrity will admit to a mistake they made, rather than blaming a colleague or trying to hide the error. Another core characteristic is \textbf{accountability}. This means taking ownership of one's responsibilities and their outcomes. A responsible programmer, for instance, doesn't just fix a bug but takes ownership of the entire resolution process, including testing and ensuring it doesn't happen again. \textbf{Discipline} is also essential, manifesting as punctuality and a commitment to meeting deadlines. An employee who consistently arrives on time for meetings and delivers their project milestones as promised demonstrates a high level of discipline. Finally, \textbf{professionalism} and \textbf{respect} tie everything together. This includes treating all colleagues, regardless of their position, with courtesy, communicating in a respectful tone (even during disagreements), and respecting company policy and property.
\end{enumerate}

\end{document}

